\section{Fixed-point Operators}\label{app:basic-alg}

In this appendix, we present algorithms from \cite{Nilsson2017} to compute the winning set for a specification in the form of \eqref{phi}. In addition to the winning set, these algorithms provide an explicit construction of the control implementation as in Definition \ref{def:cont}. 

The most general algorithm that computes the winning set and the controller for specifications in the form of \eqref{phi} is given in \eqref{win-phi}. The winning set that results from \eqref{win-phi} is equal to $ V_{\infty} $, i.e. the limit of the expanding sequence $V_k$. $ \mathcal{C} $ is the controller corresponding to the winning set $ V_{\infty} $. 

The building blocks for algorithm \eqref{win-phi} are (fixed-point based) operators \eqref{win-interm}, \eqref{win-until}, \eqref{win-pgpre},\eqref{win-inv} and \eqref{eqn:pre}. Each operator corresponds to a type of LTL formula used in \eqref{win-phi}. Note that when a fixed-point operator appears as part of a formula, it only refers to its first output, i.e. the winning set.




\begin{table}
\footnotesize
\begin{tabular}{ll}
	\begin{minipage}{0.5\textwidth}
		\begin{align}
		[V_{\infty},\mathcal{C}] =\  \text{Win}_{\exists, \forall}^{T}\left(\Square A \wedge \Diamond \Square B \wedge \left( \bigwedge_{i\in I} \Square \Diamond R^i\right)\right)\nonumber\\
		=\begin{cases}V_{\text{inv}} = \text{\text{Win}}^{T}_{\exists, \forall} ((A\mathbf{U}\emptyset)\vee \Square (A\wedge \Diamond Q))\\
		\text{Restrict~ synthesis~ to~ }V_{\text{inv}}\\
		V_0 = \emptyset,\ \mathcal{V}=\{\},\ \mathcal{K} = \{\},\  {\color{black} k=0}\\
		%{\color{purple} remove:} while\ V_{k+1} \not= V_k:\\
		{\color{black} repeat:}\\
		\ \ \ \ Z_{k+1} = \text{Pre}_{\exists,\forall}^{T,U} (V_k) \bigcup \text{PGPre}_{\exists,\forall}^{T} (V_k, Q)\\
		\ \ \ \ [V_{k+1}, \mathcal{C}_{k+1}]=\\\ \ \ \ \text{Win}_{\exists,\forall}^{T} ((B \mathbf{\ U\ }Z_{k+1}) \vee \Square (B\wedge ( \bigwedge_{i\in I}\Diamond R^i)))\\
		\ \ \ \ \mathcal{V}(k+1)=V_{k+1},\ \mathcal{K}(k+1) = \mathcal{C}_{k+1}\\
		\ \ \ \ {\color{black} k=k+1}\\
		{\color{black} until \ V_{k} = V_{k-1}}\\
		V_{\infty} = V_k,\ \mathcal{C}=(\mathcal{V},\mathcal{K},x)
		\end{cases}\label{win-phi}
		\end{align}
	\end{minipage}
	\begin{minipage}{0.5\textwidth}
		\begin{align}
		[W_{\infty},\mathcal{C}]= \ \text{Win}_{\exists,\forall}^{T}((B\mathbf{\ U\ }Z)\vee \Square (B\wedge(\bigwedge_{i\in I}\Diamond R^i)) )\nonumber\\
		=\begin{cases}
		W_0 = Q,\ \mathcal{V}=\{\},\ \mathcal{K}=\{\}\\ {\color{black} k=0}\\
		%{\color{purple} remove:} while\ W_{k+1}\not= W_k:\\
		{\color{black} repeat:}\\
		\ \ \ \ Z_{k+1}^i = Z\cup (B\cap R^i\cap \text{Pre}_{\exists, \forall}^{T,U}(W_k))\\
		\ \ \ \ [X^i, \mathcal{C}^i]= \text{Win}_{\exists,\forall}^{T} (B\mathbf{\ U\ }Z_{k+1}^i), \forall i \in I\\
		\ \ \ \ W_{k+1} = \bigcap_{i\in I} X^i\\
		\ \ \ \ {\color{black} k=k+1}\\
		{\color{black} until \ W_{k}= W_{k-1} }\\
		\mathcal{V}(1)=W_k,\\ \mathcal{V}(i+1) = B\cap R^i,\ 
		\mathcal{K}(i) = \mathcal{C}^i, \forall i\in I\\
		W_{\infty} = W_k,\ \mathcal{C} = (\mathcal{V},\mathcal{K},x)
		\end{cases}\label{win-interm}
		\end{align}
	\end{minipage} &
\end{tabular} 
\end{table}
\begin{table}
	\footnotesize
	\begin{tabular}{ll}
		\begin{minipage}{.5\textwidth}
			\begin{align}
			[X_{\infty},\mathcal{C}]= \text{Win}_{\exists,\forall}^{T} (B\mathbf{\ U\ }Z)=\quad\quad\quad\quad\quad\quad\ \ \nonumber\\
			\begin{cases}
			X_0 = \emptyset,\ \mathcal{V}=\{\},\ \mathcal{K}=\{\},\ {\color{black} k=0}\\
			%{\color{purple} remove:} while\ X_{k+1}\not= X_k\\
			{\color{black} repeat:}\\
			\ \ \ \ [V^1_k,C^1_k]=
			\text{Pre}_{\exists,\forall}^{T,U}(X_k)\\
			\ \ \ \  [V^2_k,C^2_k]=\text{PGPre}_{\exists,\forall}^{T}(Z\cup(B\cap V^1_k),B)\\
			\ \ \ \ X_{k+1} =Z\cup (B\cap V^1_k)\cup V^2_k\\
			\ \ \ \ \mathcal{V}(2k+1)=V^1_k, \mathcal{V}(2k+2)=V^2_k\\
			\ \ \ \ \mathcal{K}(2k+1)=C^1_k,\ \mathcal{K}(2k+2)=C^2_k\\
			\ \ \ \ {\color{black} k=k+1}\\
			{\color{black} until \ X_{k}= X_{k-1} }\\
			X_{\infty}=X_k,\ \mathcal{C} = (\mathcal{V},\mathcal{K},x)
			\end{cases}\label{win-until}
			\end{align}
		\end{minipage} &
		\begin{minipage}{.5\textwidth}
			\begin{align}
			[Z_{\infty},\mathcal{C}]=\text{PGPre}_{\exists,\forall}^{T} (Z,B)=\quad \ \nonumber\\
			\begin{cases}
			Z_{\infty} = Z,\ \mathcal{V} = \{\},\ \mathcal{K}=\{\}\\ k = 1\\
			for\ D\in 2^U:\\
			\ \ for\ G\in G(D):\\
			\ \ \ \ [V_k,\mathcal{C}_k]=Inv_{\exists}^{D,G}(Z_{\infty},B)\\
			\ \ \ \  Z_{\infty} = Z_{\infty} \cup V_k\\
			\ \ \ \  \mathcal{V}(k)=V_k,\ \mathcal{K}(k)=\mathcal{C}_k\\
			\ \ \ \  {\color{black} k = k+1}\\
			\mathcal{C} = (\mathcal{V},\mathcal{K},x)
			\end{cases}\label{win-pgpre}
			\end{align}
		\end{minipage} 
	\end{tabular} 
\end{table}

\begin{table}
\footnotesize
\begin{tabular}{ll}
	\begin{minipage}{.5\textwidth}
		\begin{align}
	[W,\mathcal{C}]=\text{Pre}_{\exists,\forall}^{T,U}(V)=\quad \quad \quad \quad \quad \quad \quad\quad\quad\quad\quad\nonumber\\
	\begin{cases}
	W = \{q_1\in Q: \exists (u\in U) \forall (q_2\\\quad\quad s.t.\ (q_1, u, q_2) \in \rightarrow_T), q_2\in V\}\\
	D(q)=\{u\in U: \forall (q_2\ s.t.\ (q,u,q_2)),\ q_2\in V\}\\
	\mathcal{C}=\{(q,D(q)):q\in Q,D(q)\not=\emptyset\}
	\end{cases}\label{eqn:pre}
	\end{align}
	\end{minipage}& 
\begin{minipage}{.5\textwidth}
\begin{align} [Y_{\infty},\mathcal{C}]=\text{Inv}_{\exists}^{D,G}(Z,B)=\ \quad \quad \quad \nonumber\\
\begin{cases}Y_0 = (G\cap B) - Z\\
while\ Y_{k+1}\not=Y_k:\\
\quad Y_{k+1} = Y_k\cap \text{Pre}_{\exists,\forall}^{T,D}(Y_k\cup Z)\\
[-,\mathcal{C}] = \text{Pre}^{T,D}_{\exists,\forall}(Y_k\cup Z)\\
Y_{\infty} = Y_k,\ \mathcal{C}\ restricted\ to\ Y_{\infty}
\end{cases} \label{win-inv} \end{align}
\end{minipage} 
\end{tabular} 
\end{table}

\iffalse	
\begin{algorithm}
	\small
	\caption{$ \text{PGPre}_{\exists,\forall}^{T} (Z,B) $}
	\label{alg:algorithm-label}
	\begin{algorithmic}
		\REQUIRE $ T, U, Z, B $\\
		\STATE $ Z_{\infty} = Z,\ \mathcal{V} = \{\},\ \mathcal{K}=\{\}, k = 1 $
		\FOR{$ U\in 2^U $} \FOR{$ G\in G(U) $}
		\STATE $ [V_k,\mathcal{C}_k]=Inv_{\exists}^{U,G}(Z_{\infty},B) $, $ Z_{\infty} = Z_{\infty} \cup V_k $, $ \mathcal{V}(k)=V_k,\ \mathcal{K}(k)=\mathcal{C}_k $, $ {\color{black} k = k+1} $
		\ENDFOR
		\ENDFOR
		\RETURN $ \mathcal{C} = (\mathcal{V},\mathcal{K},x) $
	\end{algorithmic}
	\label{alg: win-pgpre}
	\end{algorithm}
\fi

The winning sets {\color{black} resulting from} \eqref{win-interm}, \eqref{win-until}, \eqref{win-inv} are $ W_{\infty} $, $ X_{\infty} $ and $ Y_{\infty} $. {\color{black} $\text{Pre}_{\exists,\forall}^{T,U}$ is a one-step reachability operator, and $\text{Inv}_{\exists}^{U,G}(Z,B)$ computes $Y_\infty \subseteq (G\cap B) - Z$ from where the state can be controlled (with actions in $U$) to either remain inside $Y_\infty$ or reach $Z$, but because $G$ is a progress group under $U$, remaining indefinitely in $G$ is impossible and therefore $Z$ is eventually reached, which is why $Y_\infty$ is {\color{teal}part of the} winning set for $B\mathbf{\ U\ }Z$.} {\color{teal} $ PGPre_{\exists,\forall}^T(Z,B) $ calls $ Inv_{\exists}^{D,G}(Z,B) $  over all progress groups, to collect all the feasible states in progress gr.}
    
For controllers resulting from \eqref{win-phi}, \eqref{win-interm}, \eqref{win-until} and \eqref{win-pgpre}, their descendant controllers come from the outputs of operators they call internally. Leaf nodes in a controller always consist of simple controllers, i.e. the ones given by \eqref{eqn:pre} and \eqref{win-inv}. 

\section{Proof of Theorem 4}\label{app:pr-31}

%{\color{red} Requires careful reading}

\emph{Proof:} Assume that $ Y_t $ and $ \mathcal{C}_t $ are the winning set and controller resulting from $ \text{Inv}_{\exists}^{D,G} (\widehat{Z}) $. We want to show that $ Y_t=\widehat{Y}$, which immediately implies $ A(\mathcal{C}_t) = A(\widehat{C}) $.
	
The winning set $ Y $ of $ \text{Inv}_{\exists}^{D,G} (Z) $ is the largest subset of $ (G\cap B) - Z  $ satisfying the convergence condition w.r.t $ Z $, i.e. $ Y \subseteq Pre^{T,D}_{\exists, \forall}(Y\cup Z) $ and $Y$ is the greatest fixed-point of this operator, so is $ Y_t $ w.r.t. $ \widehat{Z} $. Also, $ Y_t \subseteq Y_0 $ for $ Y_0 $ in \eqref{patch-inv} by Theorem \ref{thm: inv-set}. 
	
	First, we show $ Y_t \subseteq \widehat{Y} $: Since $ Y_t\subseteq Y_0 $ and $ Y_k = Y_0 - \bigcup_{i\leq k} \Delta Y_i$ for $ Y_0 $, $ \Delta Y_k $ and $ Y_k $ in \eqref{patch-inv}, it is enough to show $ \Delta Y_k\cap Y_t = \emptyset,\forall k $. Proceeding by induction: $ \Delta Y_0\cap (Y_t\cup \widehat{Z}) = \emptyset $. Assume that $ \Delta Y_k \cap (Y_t\cup \widehat{Z}) = \emptyset $, then it can be easily checked that $\text{Pre}_{\forall, \exists}^{T_k,D}(\Delta Y_k) \cap  \text{Pre}_{\exists,\forall}^{T,D} (Y_t\cup \widehat{Z}) = \emptyset$ based on the fact that	 $ \rightarrow_{T_k}\subseteq \rightarrow_{T} $. For $ Y_t \subseteq \text{Pre}_{\exists,\forall}^{T,D} (Y_t\cup \widehat{Z}) $ and $ \Delta Y_{k+1} = \text{Pre}_{\forall, \exists}^{T_k,D}(\Delta Y_k) $, $ Y_t\cap \Delta Y_{k+1} = \emptyset $. Hence by induction argument, $ Y_t \cap \Delta Y_k = \emptyset, \forall k $, i.e. $ Y_t\subseteq \widehat{Y} $. 
	
Second, we show $ \widehat{Y}\subseteq Y_t $: It is enough to show that $ \widehat{Y} $ satisfies the convergence condition w.r.t. $ \widehat{Z} $. By definition of $ T_k $ in \eqref{patch-inv}, it is easy to check that $ Y_k = \{s_1: \exists u, \exists s_2\ s.t. (s_1,u,s_2)\in \rightarrow_{T_k}\} $. Therefore $ \text{Pre}_{\forall,\exists}^{T, D}(\Delta Y_k)\subseteq  Y_k$ by definition of $ \text{Pre}^{T,D}_{\forall,\exists} $ in \eqref{patch-inv}. Then we can express $ \Delta Y_{k+1}= \Delta Y_k \cup (Y_k \cap \text{Pre}_{\forall,\exists}^{T, D}(\Delta Y_k)) $. Based on the new expression of $ \Delta Y_{k+1}$ and $ Y_k \cap \widehat{Z} = \emptyset $, we have $ \Delta Y_k = Q-(Y_k\cup \widehat{Z}) $. The limit of redefined $ \Delta Y_k $ exists, for it is increasing and contained by a finite set $ Q $. Once $ \Delta Y_k $ converges, $ Y_k \cap \text{Pre}_{\forall,\exists}^{T, D}(\Delta Y_k)\subseteq \Delta Y_k $. Since $ Y_k=\widehat{Y} $ and $ \Delta Y_k $ are disjoint, $ \widehat{Y} \cap \text{Pre}_{\forall,\exists}^{T, D}(\Delta Y_k) = \emptyset $, i.e.  $ \widehat{Y} \cap \text{Pre}_{\forall,\exists}^{T, D}(Q-(\widehat{Y}\cup \widehat{Z})) = \emptyset $. It says that $ \forall s_1 \in \widehat{Y}$, not ($\forall u\in D, \exists s_2\in Q-(\widehat{Y}\cup \widehat{Z}), (s_1,u,s_2)\in \rightarrow_{T} $), i.e. $ \forall s_1 \in \widehat{Y}, \exists u\in D, \forall s_2 \in Q-(\widehat{Y}\cup \widehat{Z}),  (s_1,u,s_2)\not\in \rightarrow_{T}$ . That implies that $ \forall s_1\in \widehat{Y}, \exists u\in D, \forall (s_2\ s.t.\ (s_1,u,s_2)\rightarrow_T), s_2\in (\widehat{Y}\cup \widehat{Z}) $, that is $ \widehat{Y}\subseteq \text{Pre}_{\exists,\forall}^{T,D}(\widehat{Y}\cup \widehat{Z}) $. Thus, $ \widehat{Y} $ satisfies the convergence condition over $ \widehat{Z} $. \QEDB
%\end{proof}