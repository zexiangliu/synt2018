\section{Fixed-point Operators}\label{app:basic_alg}

In this appendix, we present algorithms from \cite{Nilsson2017} to compute the winning set for a specification in the form \eqref{phi}. In addition to the winning set, these algorithms provide an explicit construction of the control implementation as defined in Definition \ref{def:cont}. The most general algorithm is given as follows: 

\begin{align}
& [V_{\infty},\mathcal{C}] =\  \text{Win}_{\exists, \forall}^{T,U}\left(\Square A \wedge \Diamond \Square B \wedge \left( \bigwedge_{i\in I} \Square \Diamond R^i\right)\right)\\
=&\begin{cases}V_{\text{inv}} = \text{\text{Win}}^{T,U}_{\exists, \forall} ((A\mathbf{U}\emptyset)\vee \Square (A\wedge \Diamond Q))\\
\text{Restrict~ synthesis~ to~ }V_{\text{inv}}\\
V_0 = \emptyset,\ \mathcal{V}=\{\},\ \mathcal{K} = \{\}\\
{\color{purple} k=0}\\
{\color{purple} remove:} while\ V_{k+1} \not= V_k:\\
{\color{purple} repeat:}\\
\ \ \ \ Z_{k+1} = \text{Pre}_{\exists,\forall}^{T,U} (V_k) \bigcup \text{PGPre}_{\exists,\forall}^{T,U} (V_k, Q)\\
\ \ \ \ [V_{k+1}, \mathcal{C}_{k+1}]=\text{Win}_{\exists,\forall}^{T} ((B \mathbf{\ U\ }Z_{k+1}) \vee \Square (B\wedge ( \bigwedge_{i\in I}\Diamond R^i))\\
\ \ \ \ \mathcal{V}(k+1)=V_{k+1},\ \mathcal{K}(k+1) = \mathcal{C}_{k+1}\\
\ \ \ \ {\color{purple} k=k+1}\\
{\color{purple} until \ V_{k} = V_{k-1}}\\
V_{\infty} = V_k,\ \mathcal{C}=(\mathcal{V},\mathcal{K},x)
\end{cases}\label{win_phi}
\end{align}

The winning set {\color{purple} that results from} \eqref{win_phi} is {\color{purple} equal to} $ V_{\infty} $, i.e. {\color{purple} the limit of the expanding sequence $V_k$}. $ \mathcal{C} $ is the controller corresponding to the winning set $ V_{\infty} $. 

The building blocks of the above algorithm are the following fixed-point operators. Each operator corresponds to a type of LTL formula used in \eqref{win_phi}. Note that when a fixed-point operator appears as part of a formula, it only refers to its first output, i.e. the winning set.

\begin{align}
&[W_{\infty},\mathcal{C}]= \ \text{Win}_{\exists,\forall}^{T,U}((B\mathbf{\ U\ }Z)\vee \Square (B\wedge(\bigwedge_{i\in I}\Diamond R^i)) )\\
=&\begin{cases}
W_0 = Q,\ \mathcal{V}=\{\},\ \mathcal{K}=\{\}\\
{\color{purple} k=0}\\
{\color{purple} remove:} while\ W_{k+1}\not= W_k:\\
{\color{purple} repeat:}\\
\ \ \ \ Z_{k+1}^i = Z\cup (B\cap R^i\cap \text{Pre}_{\exists, \forall}^{T,U}(W_k))\\
\ \ \ \ [X^i, \mathcal{C}^i]= \text{Win}_{\exists,\forall}^{T,U} (B\mathbf{\ U\ }Z_{k+1}^i), \forall i \in I\\
\ \ \ \ W_{k+1} = \bigcap_{i\in I} X^i\\
\ \ \ \ {\color{purple} k=k+1}\\
{\color{purple} until \ W_{k}= W_{k-1} }\\
\mathcal{V}(1)=W_k,\ \mathcal{V}(i+1) = B\cap R^i,\ 
\mathcal{K}(i) = \mathcal{C}^i, \forall i\in I\\
W_{\infty} = W_k,\ \mathcal{C} = (\mathcal{V},\mathcal{K},x)
\end{cases}\label{win_interm}
\end{align}
\begin{align}
&[X_{\infty},\mathcal{C}]= \text{Win}_{\exists,\forall}^{T,U} (B\mathbf{\ U\ }Z)\\
=&\begin{cases}
X_0 = \emptyset,\ \mathcal{V}=\{\},\ \mathcal{K}=\{\}\\
while\ X_{k+1}\not= X_k\\
\ \ \ \ [V^1_k,C^1_k]=
\text{Pre}_{\exists,\forall}^{T,U}(X_k)\\
\ \ \ \  [V^2_k,C^2_k]=\text{PGPre}_{\exists,\forall}^{T,U}(Z\cup(B\cap V^1_k),B)\\
\ \ \ \ X_{k+1} =Z\cup (B\cap V^1_k)\cup V^2_k\\
\ \ \ \ \mathcal{V}(2k+1)=V^1_k, \mathcal{V}(2k+2)=V^2_k\\
\ \ \ \ \mathcal{K}(2k+1)=C^1_k,\ \mathcal{K}(2k+2)=C^2_k\\
X_{\infty}=X_k,\ \mathcal{C} = (\mathcal{V},\mathcal{K},x)
\end{cases}\label{win_until}
\end{align}
\begin{align}
&[Z_{\infty},\mathcal{C}]=\text{PGPre}_{\exists,\forall}^{T,U} (Z,B)\\
=&\begin{cases}
Z_{\infty} = Z,\ \mathcal{V} = \{\},\ \mathcal{K}=\{\}, k = 1\\
for\ U\in 2^U:\\
\ \ for\ G\in G(U):\\
\ \ \ \ \ \  [V_k,\mathcal{C}_k]=Inv_{\exists}^{U,G}(Z_{\infty},B)\\
\ \ \ \ \ \ Z_{\infty} = Z_{\infty} \cup V_k\\
\ \ \ \ \ \ \mathcal{V}(k)=V_k,\ \mathcal{K}(k)=\mathcal{C}_k,\ k++\\
\mathcal{C} = (\mathcal{V},\mathcal{K},x)
\end{cases}\label{win_pgpre}
\end{align}
\begin{align}
&[W,\mathcal{C}]=\text{Pre}_{\exists,\forall}^{T,U}(V)\\
=&\begin{cases}
W = \{q_1\in Q: \exists (u\in U) \forall (q_2\ s.t.\ (q_1, u, q_2) \in \rightarrow_T), q_2\in V\}\\
D(q)=\{u\in U: \forall (q_2\ s.t.\ (q,u,q_2)),\ q_2\in V\}\\
\mathcal{C}=\{(q,D(q)):q\in Q, D(q)\not=\emptyset\}
\end{cases}\label{eqn:pre}\\
&[Y_{\infty},\mathcal{C}]=\text{Inv}_{\exists}^{D,G}(Z,B)\\
=&\begin{cases}Y_0 = (G\cap B) - Z\\
while\ Y_{k+1}\not=Y_k:\\
\ \ \ \ Y_{k+1} = Y_k\cap \text{Pre}_{\exists,\forall}^{T,D}(Y_k\cup Z)\\
[-,\mathcal{C}] = Pre^{T,D}_{\exists,\forall}(Y_k\cup Z)\\
Y_{\infty} = Y_k,\ \mathcal{C}\ restricted\ to\ Y_{\infty}
\end{cases} \label{win_inv}
\end{align}

The winning sets {\color{purple} resulting from} \eqref{win_interm}, \eqref{win_until}, \eqref{win_inv} are $ W_{\infty} $, $ X_{\infty} $ and $ Y_{\infty} $. {\color{purple} $\text{Pre}_{\exists,\forall}^{T,U}$ is a one-step reachability operator, and $\text{Inv}_{\exists}^{U,G}(Z,B)$ computes $Y_\infty \subseteq (G\cap B) - Z$ from where the state can be controlled (with actions in $U$) to either remain inside $Y_\infty$ or reach $Z$, but because $G$ is a progress group under $U$, remaining indefinitely in $G$ is impossible and therefore $Z$ is
	eventually reached, which is why $Y_\infty$ is a winning set for $B\mathbf{\ U\ }Z$.}
    
    For \eqref{win_phi}, \eqref{win_interm}, \eqref{win_until} and \eqref{win_inv}, the number of sub-controllers in each controller is proportional 
    {\color{purple} the number of children controllers is equal} 
    to the number of iterations before the corresponding fixed-point operator converges. Simple controllers are smallest elements in a controller, i.e. the ones given by \eqref{eqn:pre} and \eqref{win_inv}. 
    
    \section{Proof of Theorem 3.4}\label{app:pr_31}
    
    {\color{red} Requires careful reading}

\begin{proof}
	Assume that $ Y_t $ and $ \mathcal{C}_t $ are the winning set and controller resulting from $ \text{Inv}_{\exists}^{D,G} (\widehat{Z}) $. Here we are going to prove that $ Y_t=\widehat{Y}$. Once we have $ Y_t=\widehat{Y}$, it is easy to check that $ T(\mathcal{C}_t) = T(\widehat{C}) $.
	
	By \eqref{win_inv}, the winning set $ Y $ of $ \text{Inv}_{\exists}^{D,G} (Z) $ is the largest subset of $ (G\cap B) - Z  $ satisfying the convergence condition w.r.t $ Z $, i.e. $ Y \subseteq Pre^{T,D}_{\exists, \forall}(Y\cup Z) $, so is $ Y_t $ w.r.t. $ \widehat{Z} $.
	
	First, prove that $ Y_t \subseteq Y_0 $ for $ Y_0 $ in \eqref{patch_inv}: Check that $ V= Y_t \cap (G\cap B -Z) $ is a subset of $ G\cap B -Z $ satisfying the convergence condition over $ Z $. For $ V_1, V_2 \subseteq G\cap B-Z $ satisfying convergence condition, $ V_1\cup V_2 $ satisfies convergence condition too, which implies $ V \subseteq Y $ (otherwise, $ Y\subseteq Y\cup V $ is a larger subset of $ G\cap B-Z $ satisfying the convergence condition than $ Y $. Contradiction!) So $ Y_t = V\cup (Y_t\cap \Delta Z) \subseteq Y\cup (\Delta Z \cap G\cap B) = Y_0$. 
	
	Second, prove $ Y_t \subseteq \widehat{Y} $: Since $ Y_t\subseteq Y_0 $ and $ Y_k = Y_0 - \bigcup_{i\leq k} \Delta Y_i$, it's enough to show $ \Delta Y_k\cap Y_t = \emptyset,\forall k $. Prove by induction: $ \Delta Y_0\cap (Y_t\cup \widehat{Z}) = \emptyset $. Assume that $ \Delta Y_k \cap (Y_t\cup \widehat{Z}) = \emptyset $, then $\text{Pre}_{\forall, \exists}^{T_k,D}(\Delta Y_k) \cap  \text{Pre}_{\exists,\forall}^{T,D} (Y_t\cup \widehat{Z}) = \emptyset$ by \eqref{eqn:pre} and the fact that $ \rightarrow_{T_k}\subseteq \rightarrow_{T} $. For $ Y_t \subseteq \text{Pre}_{\exists,\forall}^{T,D} (Y_t\cup \widehat{Z}) $ and $ \Delta Y_{k+1} = \text{Pre}_{\forall, \exists}^{T_k,D}(\Delta Y_k) $, $ Y_t\cap \Delta Y_{k+1} = \emptyset $. Hence by induction argument, $ Y_t \cap \Delta Y_k = \emptyset, \forall k $, i.e. $ Y_t\subseteq \widehat{Y} $. 
	
	Finally, prove $ \widehat{Y}\subseteq Y_t $: It's enough to show that $ \widehat{Y} $ satisfies the convergence condition over $ \widehat{Z} $. By definitions of $ T_k $, $ \Delta Y_{k+1}= Y_k \cap \text{Pre}_{\forall,\exists}^{T, D}(\Delta Y_k)$. Then by the additivity of $ \text{Pre}^{T,D}_{\forall,\exists} $ (i.e. for any $ X_1 $ and $ X_2 $, $ \text{Pre}_{\forall,\exists}^{T,D} (X_1\cup X_2)=\text{Pre}_{\forall,\exists}^{T,D} (X_1)\cup \text{Pre}_{\forall,\exists}^{T,D} (X_2) $), it's easy to check that redefining fifth line in \eqref{patch_inv} by $ \Delta Y_{k+1} = \Delta Y_k \cup (Y_k \cap \text{Pre}_{\forall,\exists}^{T, D}(\Delta Y_k)) $ doesn't change $ Y_{k+1} $.  By the new definition of $ \Delta Y_k $ and $ Y_k \cap \widehat{Z} = \emptyset $, we have $ \Delta Y_k = Q-(Y_k\cup \widehat{Z}) $. The limit of redefined $ \Delta Y_k $ exists, for it is increasing and contained by a finite set $ Q $. Once $ \Delta Y_k $ converges, $ Y_k \cap \text{Pre}_{\forall,\exists}^{T, D}(\Delta Y_k)\subseteq \Delta Y_k $. Since $ Y_k=\widehat{Y} $ and $ \Delta Y_k $ are disjoint, $ \widehat{Y} \cap \text{Pre}_{\forall,\exists}^{T, D}(\Delta Y_k) = \emptyset $, i.e.  $ \widehat{Y} \cap \text{Pre}_{\forall,\exists}^{T, D}(Q-(\widehat{Y}\cup \widehat{Z})) = \emptyset $. It says that $ \forall s_1 \in \widehat{Y}$, not $\forall u\in D, \exists s_2\in Q, (s1,u,s2)\in \rightarrow_{T} $ and $s_2\in (Q-(\widehat{Y}\cup \widehat{Z}))$, i.e. $ \forall s_1 \in \widehat{Y}, \exists u\in D, \forall s_2 \in Q,  (s_1,u,s_2)\not\in \rightarrow_{T}$ or $ s_2\in (\widehat{Y}\cup \widehat{Z})$. That implies that $ \widehat{Y}\subseteq \text{Pre}_{\exists,\forall}^{T,D}(\widehat{Y}\cup \widehat{Z}) $, i.e. $ \widehat{Y} $ satisfies the convergence condition over $ \widehat{Z} $.
\end{proof}